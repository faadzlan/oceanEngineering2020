%% 
%% Copyright 2019 Elsevier Ltd
%% 
%% This file is part of the 'CAS Bundle'.
%% --------------------------------------
%% 
%% It may be distributed under the conditions of the LaTeX Project Public
%% License, either version 1.2 of this license or (at your option) any
%% later version.  The latest version of this license is in
%%    http://www.latex-project.org/lppl.txt
%% and version 1.2 or later is part of all distributions of LaTeX
%% version 1999/12/01 or later.
%% 
%% The list of all files belonging to the 'CAS Bundle' is
%% given in the file `manifest.txt'.
%% 
%% Template article for cas-dc documentclass for 
%% double column output.

%\documentclass[a4paper,fleqn,longmktitle]{cas-dc}
\documentclass[a4paper,fleqn]{cas-sc}

%\usepackage[authoryear,longnamesfirst]{natbib}
\usepackage[authoryear]{natbib}
\usepackage{amsmath}
\usepackage{gensymb}
\usepackage{subcaption}
\usepackage{siunitx}
\usepackage{setspace}
\usepackage{lineno}
%\usepackage[numbers]{natbib}

%%%Author definitions
\def\tsc#1{\csdef{#1}{\textsc{\lowercase{#1}}\xspace}}
\tsc{WGM}
\tsc{QE}
\tsc{EP}
\tsc{PMS}
\tsc{BEC}
\tsc{DE}
%%%

\linenumbers

\begin{document}


\let\WriteBookmarks\relax
\def\floatpagepagefraction{1}
\def\textpagefraction{.001}
\shorttitle{Lift Evolution in a Pure Cruciform Energy Harvester}
\shortauthors{A. Adzlan, M.S.M. Ali and S.A.Z.S. Salim}

\title [mode = title]{Temporal Evolution of Lift in a Pure Cruciform System for Energy Harvesting}                      
%\tnotemark[1,2]
%
%\tnotetext[1]{This document is the results of the research
%   project funded by the National Science Foundation.}
%
%\tnotetext[2]{The second title footnote which is a longer text matter
%   to fill through the whole text width and overflow into
%   another line in the footnotes area of the first page.}



\author[1,2]{Ahmad Adzlan}[orcid=0000-0003-0290-3185]
\cormark[1]
%\fnmark[1]
\ead{aafkhairi@graduate.utm.my}
%\ead[url]{www.cvr.cc, cvr@sayahna.org}

%\credit{Conceptualization of this study, Methodology, Software}

\address[1]{Malaysia-Japan International Institute of Technology, Universiti Teknologi Malaysia, 54200 Kuala Lumpur, Malaysia}

\author[1]{Mohamed Sukri Mat Ali}
%\fnmark[1]
\ead{sukri.kl@utm.my}
%\ead[URL]{www.sayahna.org}
%\credit{Data curation, Writing - Original draft preparation}

\author[1]{Sheikh Ahmad Zaki Shaikh Salim}
%\fnmark[1]
\ead{sheikh.kl@utm.my}
%\ead[URL]{www.sayahna.org}
%\credit{Data curation, Writing - Original draft preparation}

\address[2]{Faculty of Engineering, Universiti Malaysia Sarawak, 94300 Kota Samarahan, Sarawak, Malaysia}

\cortext[cor1]{Corresponding author}
%\cortext[cor2]{Principal corresponding author}
%\fntext[fn1]{This is the first author footnote. but is common to third
%  author as well.}
%\fntext[fn2]{Another author footnote, this is a very long footnote and
%  it should be a really long footnote. But this footnote is not yet
%  sufficiently long enough to make two lines of footnote text.}

%\nonumnote{This note has no numbers. In this work we demonstrate $a_b$
%  the formation Y\_1 of a new type of polariton on the interface
%  between a cuprous oxide slab and a polystyrene micro-sphere placed
%  on the slab.
%  }

%The macros for commonly used symbols
\newcommand{\ypl}{y^{+}} %yPlus
\newcommand{\ured}{U^{*}} %reduced velocity
\newcommand{\yrms}{y^{*}_{\text{RMS}}} %root-mean-square of the normalised cylinder displacement
\newcommand{\ystr}{y^{*}} %the normalised cylinder displacement
\newcommand{\fstr}{f^{*}} %the normalised vibration frequency
\newcommand{\fn}{f_{n}} %system natural frequency
\newcommand{\fcyl}{f_{\text{cyl.}}} %frequency of cylinder vibration
\newcommand{\fosc}{f_{\text{osc.}}} %frequency of cylinder oscillation
\newcommand{\fclstr}{f_{\text{Cl}}^{*}} %normalised frequency of lift coefficient
\newcommand{\flrms}{F_{\text{L,RMS}}} %root-mean-square of the lift force
\newcommand{\fl}{F_{\text{L}}} %the lift force
\newcommand{\clrms}{\text{Cl}_{\text{RMS}}} %root-mean-square of the lift coefficient
\newcommand{\cflyt}{C_{F_{L},y(t)}} %IMF component of lift that is most similar to the displacement signal in terms of temporal evolution of amplitude and frequency, differing only perhaps in phase OR the component of lift with the highest correlation to the displacement signal
\newcommand{\ccli}{C_{\text{Cl},i}} %the ith component of lift coefficient
\newcommand{\cclystr}{C_{\text{Cl},\ystr}} %the ith component of lift coefficient
\newcommand{\cflm}{C_{F_{L},\text{max}}} %IMF component of lift that has maximum RMS amplitude in the IMF set
\newcommand{\cyrms}{C_{y,\text{RMS}}} %the RMS of the component of lift that is most correlated with the cylinder displacement signal
\newcommand{\cclrms}{C_{\text{Cl},\text{RMS}}} %the RMS of the component of lift that is most correlated with the cylinder displacement signal (new symbol)
\newcommand{\afl}{\alpha_{F_{L}}} %ratio between two dominant IMF components of the lift
\newcommand{\pfrms}{P_{\text{Fluid,RMS}}} %estimated root-mean-square of fluid power
\newcommand{\pmrms}{P_{\text{Mech.,RMS}}} %estimated root-mean-square of mechanical power
\newcommand{\re}{\text{Re}} %Reynolds number
\newcommand{\st}{\text{St}} %Strouhal number
\newcommand{\phim}{\phi_{\text{mean}}} %mean phase lag
\newcommand{\wcl}{W_{\text{cyl.}}} %mean work done by cylinder over one cycle of vibration
\newcommand{\tosc}{T_{\text{osc.}}} %mean period of cylinder oscillation
\newcommand{\meff}{m_{\text{eff.}}} %effective mass
\newcommand{\zetatot}{\zeta_{tot.}} %total damping of the system

%Macros that are shorthands in writing
\newcommand{\rms}{root-mean-square} %shorthand for root-mean-square

%Macros used in writing section on GCI study
\newcommand{\rp}{r^{p}} %refinement ratio, used in GCI study
\newcommand{\fre}{f_{\text{RE}}} %Richardson extrapolation of quantity of interest, used in GCI study

%The macros for freestream velocities
\newcommand{\uon}{\unit{0.1}{\metre\per\second}}
\newcommand{\utw}{\unit{0.2}{\metre\per\second}}
\newcommand{\uth}{\unit{0.3}{\metre\per\second}}
\newcommand{\ufo}{\unit{0.4}{\metre\per\second}}
\newcommand{\ufi}{\unit{0.5}{\metre\per\second}}
\newcommand{\usi}{\unit{0.6}{\metre\per\second}}
\newcommand{\use}{\unit{0.7}{\metre\per\second}}
\newcommand{\uei}{\unit{0.8}{\metre\per\second}}
\newcommand{\uni}{\unit{0.9}{\metre\per\second}}
\newcommand{\ute}{\unit{1.0}{\metre\per\second}}
\newcommand{\uel}{\unit{1.1}{\metre\per\second}}
\newcommand{\utv}{\unit{1.2}{\metre\per\second}}
\newcommand{\utt}{\unit{1.3}{\metre\per\second}}

\newcommand{\uron}{2.3}
\newcommand{\urtw}{4.5}
\newcommand{\urth}{6.8}
\newcommand{\urfo}{9.1}
\newcommand{\urfi}{11.4}
\newcommand{\ursi}{13.6}
\newcommand{\urse}{15.9}
\newcommand{\urei}{18.2}
\newcommand{\urni}{20.5}
\newcommand{\urte}{22.7}
\newcommand{\urel}{25.0}
\newcommand{\urtv}{27.3}
\newcommand{\urtt}{29.5}


\begin{abstract}
We investigated the amplitude and frequency responses of a circular cylinder - strip plate cruciform system in the Reynolds number range $1.1\times10^{3}<Re<14.6\times10^{3}$ numerically using the open source C++ library: OpenFOAM. We decomposed the cylinder displacement and lift time series from our numerical results into Hilbert transform-friendly signals using the ensemble empirical mode decomposition (EEMD) method. The mean phase lag obtained through Hilbert transform points to the existence of an initial branch-like state, with a phase lag of $\approx20$ deg. in the narrow region of $\ured$ close to 18.2. Then, the mean phase lag jumps from $\approx20$ deg. to $\approx110$ deg. once $\ured$ reaches 20.5, analogous to the transition to upper branch in a KVIV amplitude response curve. The instantaneous phase lag shows that SVIV is quasi-periodic up until $\ured = 27.3$. Between $18.2<\ured<22.5$, Karman vortex shedding contributes nearly as much as streamwise vortex shedding to the \rms{} amplitude of total lift, while between $25.0\leq\ured\leq29.5$, the Karman component contribution is on average twice that of the streamwise component. These findings hint at the possibility to improve the power output of the harvester by a factor of two between $18.28<\ured<22.5$ and by a factor of three between $25.0 \leq \ured \leq 29.5$, if we can unite the contribution to the \rms{} amplitude of the total lift under a single vibration-driving mechanism: the shedding of streamwise vortex.
\end{abstract}

\begin{graphicalabstract}
  \includegraphics[width=1\textwidth]{figs/graphicalAbstract}
\end{graphicalabstract}

\begin{highlights}
\item Decomposition of the lift coefficient signal via ensemble empirical mode decomposition (EEMD) brings to the fore the components of lift generated by the shedding of Karman and streamwise vortices, which in its original form is observed as one non-monotonic lift signal in the streamwise vortex-induced vibration (SVIV) regime.
\item Determination of phase lag between lift and cylinder displacement using Hilbert-Huang Transform (HHT) reveals evidence suggesting the existence of an initial branch for SVIV.
\item Contribution to the total \rms{} (RMS) lift amplitude from the shedding of both Karman and streamwise vortices suggest that we might be able to enlarge the RMS amplitude -- and as a result harnessed power -- if we can redirect energy away from Karman vortices towards streamwise vortices, in the SVIV regime.
\end{highlights}

\begin{keywords}
  Vortex-induced vibration \sep Vibration energy harvester \sep CFD simulation \sep Streamwise vorticity \sep Ensemble empirical mode decomposition (EEMD) \sep Hilbert transform
\end{keywords}


\maketitle

\doublespacing

\section{Introduction} \label{sec:intro}
Streamwise vortex-induced vibration (SVIV) is a type of vortex-induced vibration (VIV) driven by vortical structures whose vorticity vector points in the direction of the free stream. In recent decades, there have been efforts to exploit the SVIV phenomenon from cruciform structures for energy harvesting. The literature on this subject can be broadly categorised into two groups: how the mechanical properties of the oscillator (e.g., mass ratio, damping, etc.) affects the amplitude/frequency response of SVIV \citep{Koide2009,Koide2013,Nguyen2012} and how the minutiae of the flow field affect the force driving the vibration of the cylinder, i.e. the fluid mechanical aspect of the system \citep{Deng2007,Koide2017,Zhao2018a}.

In the first focus area, researchers studied some permutation of the following method to convert the vibration into electrical power. The method consists of a coil and magnet. The coil, which moves with the vibrating cylinder, creates relative motion against the magnet, which is placed in the hollow of the coil \citep{Koide2009}. While investigating the system at a Reynolds number in the order of $\re \sim O \left( 10^{4} \right)$, \citet{Koide2009} showed that increased damping due to energy harvesting reduces the maximum vibration amplitude close to a factor of 4. Amplitude reduction due to increased total damping was also mentioned in \citet{Bernitsas2008a,Bernitsas2008b,Bernitsas2009}. Further investigation in \citet{Nguyen2012} revealed that damping not only affects the amplitude response of the cylinder but also narrows the synchronisation region between vortex shedding and cylinder vibration. Moreover, \citet{Nguyen2012} demonstrated a strong coupling between mass ratio and damping in determining both the width of the synchronisation region and the maximum amplitude response of the cylinder.

In the second focus area, investigators turned their attention to the details of the flow where streamwise vortex shedding occurs. One such study carefully shot motion pictures of the dye-injected flow \citep{Koide2017} at Reynolds number in the order of $\re \sim O\left( 10^{3} \right)$. A lower Reynolds number (Re) reduces the amount of turbulence in the flow, allowing a clearer shot of the vortex structures. Their study also highlights the higher level of turbulence produced by the circular cylinder---strip-plate cruciform in contrast to the twin circular cylinder cruciform, which diminishes the periodicity of vortex shedding. Although visually enlightening, this and other more qualitative studies contribute little towards improving our understanding of the relationship between vortex shedding and the resulting lift. \citet{Deng2007} demonstrated a way to overcome such a shortcoming.

In their study, \citet{Deng2007} examined the flow field of a twin circular cylinder cruciform using computational fluid dynamics (CFD). Their domain stretches  $28D$  in the streamwise direction,  $16D$  in the transverse direction and  $12D$  in the spanwise direction. They studied an Re range yet another order of magnitude smaller than that studied by \citet{Koide2017}, possibly to get an even clearer visualisation of the vortical structures with less turbulence, and to ease computational requisites.

At a fixed  $\re = 150$ , streamwise vortices form even at a gap ratio of $2$. This result differs quite strikingly from \citet{Koide2006,Koide2007}, conducted at an Re twice the order of magnitude of \citet{Deng2007}, an indication that the minimum gap ratio needed for the onset of streamwise varies with respect to Re.

They also observed that when the gap ratio $G$, which they denote as  $L/D$  in their paper, increases from 3 to 4, the maximum amplitude of the lift coefficient increases by almost threefold. This can be attributed quite easily to the current vortex pair shed by the upstream cylinder. The downstream cylinder immediately disturbs the pair shed from the upstream cylinder when  $G=3$. The lift coefficient increases by about a factor of 3 when this immediate disturbance diminishes at  $G=4$. The visualisation of three-dimensional (3D) vorticity isocontours enables us to quickly establish this link vis-\`{a}-vis the lift coefficient signal. The authors use of CFD made this possible.

A similar study in the order of magnitude $\re \sim O \left( 10^{2} \right)$ by \citet{Zhao2018a} particularly highlighted the immense utility of CFD as a tool to research SVIV or flow around a cruciform in general. They computed the sectional lift coefficient along the upstream cylinder and the time history of this sectional lift coefficient points towards two different modes of vortex shedding, namely, parallel and K-shaped. They also paid attention to the local flow patterns that vary along the length of the upstream cylinder such as the trailing vortex flow, necklace vortex flow and flow in the small gap (denoted as SG flow). As shown by the discontinuities in the phase angle of the sectional lift coefficient along the upstream cylinder, we wondered whether the lift coefficient here can be considered due to streamwise vortex shedding alone, when Karman vortex streamlines were also observed some distance away from the junction of the cruciform.

While we find in persuasive to attribute the frequency/ amplitude modulation of the lift signal in a KVIV system to the shedding of Karman vortices, one becomes more cautious in doing to the same solely as the result of streamwise vortex shedding in an SVIV system. The main reason behind this lies in the fact that the cylinder continues to shed Karman vortices even after the onset for streamwise vortex shedding and SVIV \citep{Shirakashi1989}. This point leads us to hypothesise that the lift signal is more appropriately viewed as the streamwise-Karman vortex-induced composite lift signal. However, we could not find studies that took this viewpoint in their investigation of SVIV and worked out its implication on power generation.

The objectives of this study are thus threefold: (1) to take a closer look at the amplitude and frequency response of a circular cylinder-strip plate cruciform, especially in $\ured$ ranges where the transition from KVIV to SVIV occurs, (2) to demonstrate the compositeness of the lift signal of an SVIV system and establish the difference between the lift signal characteristics in the KVIV and SVIV regime and (3) to shed light on how the contribution from the Karman and streamwise components of lift changes as we increase $\ured$ after the onset of SVIV and predict how much improvement in the power generation can be anticipated if we are able to unify the lift amplitude contributions due to Karman and streamwise vortex shedding. The following \S\ref{sec:method} details the methodology we employ to conduct this study. We present and discuss our results in \S\ref{sec:singPlateResp}, \S\ref{sec:tempEvo}, and \S\ref{sec:estimPow}. We describe our conclusions in \S\ref{sec:conclusions}.

\begin{figure}
  \centering
  \includegraphics[width=0.5\textwidth]{figs/figure1}
  \caption{A schematic of the circular cylinder-strip plate cruciform system. Alternate shedding of the steamwise vortices create the alternating lift that drives the vibration of the cylinder.}
  \label{fig:cruciformSystemSchematic}
\end{figure}

\section{Methodology} \label{sec:method}
\subsection{Problem geometry} \label{ssec:probGeo}
The geometrical setup for this study builds on the work of \citet{Maruai2017,Maruai2018} who studied both experimentaly and numerically the FIM of a square cylinder with a downsteam flate plate. Their simulation results are in good agreement with their own experiment, and with the experimental results of \citet{Kawabata2013}, in the Reynolds number range $3.6\times10^{3}<\text{Re}<12.5\times10^{3}$. This is well within the Reynolds number studied in this work, i.e. $1.1\times10^{3}<\text{Re}<14.6\times10^{3}$.

Our $x-y$ plane fundamentally follows the dimensions used in \citet{Maruai2017,Maruai2018}, except for the cylinder shape, which in this study is circular, and the $20D$ distance to the outlet is measured from the downstream face of the strip-plate. This is shown in Fig. \ref{fig:problemGeometry}. We chose the cylinder-plate gap $G$ to be $0.26D$, as previous works have shown this gap size sustains the highest SVIV amplitude over the widest range of $\ured$, in comparison to other gap sizes.

\begin{figure}
  \centering
  \includegraphics[width=0.5\textwidth]{figs/figure2}
  \caption{Problem geometry and coordinate system used. (a) shows the side view of the simulation domain (viewed parallel to the freestream) while (b) shows the top view of the simulation domain (viewed perpendicular to the freestream). Note that the gap ratio $G$ between the cylinder and the strip plate is $0.16D$, and the ACMI patch is located midway through the gap, i.e., $0.08D$ downstream from the trailing edge of the cylinder.}
  \label{fig:problemGeometry}
\end{figure}

As the problem geometry is explicitly three-dimensional (3D), the $x-y$ plane is extruded in the $z$ direction, thus obtaining a 3D domain. As can be seen in Fig. \ref{fig:problemGeometry}, the circular cylinder extends from $z/D=7.5$ to $z/D=-7.5$, while the strip-plate extends from $-10.5$ to $y/D=10.5$. The $z-$direction extent is set as $z/D=\pm7.5$ is already more than twice the spanwise reach of the streamwise vortex, thus sufficient for the vortices to materialise in our numerical solution. To compare, the spanwise extent of the numerical study by \citet{Deng2007}, is $z/D=\pm6$ and the spanwise extents of experiments by \citet{Nguyen2012} and \citet{Koide2013} are $z/D=\pm5$.

\subsection{Numerical method} \label{ssec:numMeth}
The objectives of our study necessitate the solution of the continuity, and 3D unsteady Reynolds averaged Navier-Stokes (3D URANS) equations. We achieve this by using OpenFOAM, an open-source computational fluid dynamics (CFD) platform written in C++. Specifically, we work to solve the following continuity and URANS equations.

\begin{equation}
  \frac{\partial U_{i}}{\partial x_{i}}=0,
  \label{eq:continuity}
\end{equation}

\begin{equation}
  \frac{\partial U_{i}}{\partial t}+U_{j}\frac{\partial U_{i}}{\partial x_{j}} = -\frac{1}{p}\frac{P}{x_{i}}+\frac{\partial}{\partial x_{j}} \left( 2\nu S_{ij}-\overline{u'_{j}u'_{i}} \right).
  \label{eq:navier-stokes}
\end{equation}

The symbols $U$, $x$, $t$, $\rho$, $P$, $\nu$, $S$, and $u'$ are the mean component of velocity, spatial component, time density, pressure, kinematic viscosity, mean strain rate and the fluctuating component of velocity, respectively. The mean strain rate $S_{ij}$ is given by

\begin{equation}
  S_{ij} = \frac{1}{2} \left( \frac{\partial U_{i}}{\partial x_{j}} + \frac{\partial U_{j}}{\partial x_{i}} \right).
  \label{eq:sij}
\end{equation}

This study employs the Spalart-Allmaras turbulence model to approximate the Reynolds stress tensor $\tau_{ij} = \overline{u'_{j}u'{i}}$. This turbulence model has been shown to produce results that agree reasonably well with experiments in similar flow-induced motion (FIM) studies \citep{Ding2015a,Ding2015b}. We use the Boussinesq approximation to relate the Reynolds stress tensor to the mean velocity gradient

\begin{equation}
  \tau_{ij} = 2 \nu_{T}S_{ij},
  \label{eq:tauij}
\end{equation}

\noindent where $\nu_{T}$ represents the kinetic eddy viscosity. $\nu_{T}$ is, in turn, a function of $\tilde{\nu}$ and $f_{\nu 1}$, while $f_{\nu 1}$ is a function of $\chi$ and $c_{\nu 1}$, and $\chi$ a function of $\tilde{\nu}$ and $\nu$, as shown in Eq. \ref{eq:kineticeddy}.

\begin{subequations}
  \label{eq:kineticeddy}
  \begin{align}
    \nu_{T}   & = \tilde{\nu} f_{\nu 1}, \label{eq:kineticeddyA}\\
    f_{\nu 1} & = \frac{\chi^{3}}{\chi^{3}+c^{3}_{\nu 1}}, \label{eq:kineticeddyB}\\
    \chi      & = \frac{\tilde{\nu}}{\nu}. \label{eq:kineticeddyC}
\end{align}
\end{subequations}

\noindent Here, $\tilde{\nu}$ serves to mediate the turbulence model and dictates how $\tilde{\nu}$ is conserved.

\begin{align}
  \label{eq:kineticEddyTransport}
  \frac{\partial \tilde{\nu}}{\partial t} &+ U_{j} \frac{\partial \tilde{\nu}}{\partial x_{j}} = c_{b1}\tilde{S}\tilde{\nu} - c_{w1} f_{w} \left( \frac{\tilde{\nu}}{D} \right)^{2} \nonumber \\
  &\qquad {} + \frac{1}{\sigma} \left\{ \frac{\partial}{\partial x_{j}} \left[ \left( \nu + \tilde{\nu} \right) \frac{\partial \tilde{\nu}}{\partial x_{j}} \right] c_{b2} \frac{\partial \tilde{\nu}}{\partial x_{i}} \frac{\partial \tilde{\nu}}{\partial x_{i}} \right\}
\end{align}

$c_{b1}$, $c_{b2}$, and $c_{\nu 1}$ are constant with values $0.1335$, $0.622$ and $7.1$ respectively. $c_{w1}$ is given by

\begin{equation}
  c_{w1} = \frac{c_{b1}}{\kappa} + \frac{1+c_{b2}}{\sigma},
  \label{eq:cw1}
\end{equation}

\noindent where additional constants $\kappa$ and $\sigma$ are $0.41$ and $2/3$ respectively. $f_{w}$, on the other hand, is given by

\begin{equation}
  f_{w} = g \left( \frac{1 + c^{6}_{w3}}{g^{6} + c_{w3}} \right)^{\frac{1}{6}}.
  \label{eq:fw}
\end{equation}

\noindent Here, $c_{w3} = 2$ while $g$ is given by

\begin{equation}
  g = r + c_{w2} \left( r^{6} - r \right),
  \label{eq:g}
\end{equation}

\noindent where $r$ is

\begin{equation}
  r = \text{min} \left( \frac{\tilde{\nu}}{\tilde{S} \kappa^{2} d^{2}}, 10 \right),
  \label{eq:r}
\end{equation}

Additionally, $\tilde{S}$ is

\begin{equation}
  \tilde{S} = \Omega + \frac{\tilde{\nu}}{\kappa^{2} d^{2}} f_{\nu 2},
  \label{eq:sTilde}
\end{equation}

\noindent where $\Omega$ and $d$ are the magnitude of vorticity and the distance from the mesh nodes to the nearest wall, respectively. Finally, $f_{\nu 2}$ is

\begin{equation}
  f_{\nu 2} = 1 - \frac{\chi}{1 + \chi f_{\nu 1}}.
  \label{eq:fv2}
\end{equation}

\noindent We solve these equations numerically using the PIMPLE algorithm, which combines the transient solver PISO with the steady-state solver SIMPLE for improved numerical stability.

\subsection{Dynamic mesh motion} \label{ssec:dynMesh}

In this study, the cylinder in VIV moves perpendicular to the free stream direction. The motion unavoidably distorts the mesh around it, degrading important mesh metrics such as non-orthogonality and skewness. However, we can diffuse the mesh deformation to the neighbouring nodes as per the following Laplace equation,

\begin{equation}
  \nabla \cdot \left( \gamma \nabla u \right) = 0.
  \label{eq:laplace}
\end{equation}

\noindent Here, $u$ represents the mesh deformation velocity and $\gamma$ is displacement diffusion. We chose $\gamma = 1/l^{2}$, where $l$ is the cell centre distance to the nearest cylinder edges. We implement the GAMG linear solver with the Gauss-Seidel smoother to solve Eq. \ref{eq:laplace}. The dynamic mesh algorithm then updates the mesh node positions according to the following equation.

\begin{equation}
  x_{\text{new}} = x_{\text{old}} + u \Delta t
  \label{eq:meshNodeUpdate}
\end{equation}

\noindent The solver resumes the solution of Eqs. \ref{eq:continuity} and \ref{eq:navier-stokes} once the mesh node positions are updated.

Another dynamic mesh handling technique used in this study is the arbitrarily coupled mesh interface (ACMI) that allows non-conforming meshes to slide over another, thus preserving the mesh quality around a moving object. The tiny gap between the cylinder and strip-plate, limits our ability to diffuse the mesh deformation to the surrounding space. ACMI is thus implemented at the centre of the gap between the circular cylinder and the strip-plate, as shown in Fig. \ref{fig:problemGeometry}, to circumvent this problem. This method has been successfully implemented in the works of \citet{Ding2015b,Zhang2018}, preserving the quality of their mesh and controlling their Courant-Friedrichs-Lewy (CFL) number.

\subsection{Open flow chanel experiment} \label{ssec:openFlowExp}

We set up an experimental rig to validate our numerical results in the vicinity of reduced velocity $\ured = 22.7$. Here, $\ured = U/\fn D$, with $U$, $\fn$ and $D$ being the freestream velocity, natural frequency of the system and the diameter of the circular cylinder respectively. We chose $\ured = 22.7$ because that value of $\ured$ is where the vibration-driving mechanism is known to transit from Karman to streamwise vortex shedding \citep{Koide2013}. The experimental rig consists of a closed-loop open channel circuit based on the water tunnel used by \citet{Nguyen2012}, shown in Fig. \ref{fig:experimentalSetup}. The cross-section of our test section is a square with sides $100$ mm in length. The test section is $1500$ mm long.

\begin{figure}
  \centering
  \begin{subfigure}[h]{0.5\textwidth}
    \includegraphics[width=\textwidth]{figs/figure3a}
    \caption{}
    \label{fig:rigSketch}
  \end{subfigure}

  \begin{subfigure}[h]{0.35\textwidth}
    \includegraphics[width=\textwidth]{figs/figure3b}
    \caption{}
    \label{fig:damperSketch}
  \end{subfigure}

  \caption{A schematic of our experimental setup. Fig. \ref{fig:rigSketch} presents a 3D schematic of the experimental rig while Fig. \ref{fig:damperSketch} shows an enlarged schematic of the damping system.} \label{fig:experimentalSetup}
\end{figure}

The system for providing elastic support and damping to the circular cylinder follows closely those used by \citet{Kawabata2013} and \citet{Koide2013,Koide2017}, which can be summarised as follows. The stiffness coefficient $k$ of the plate spring is determined through a simple weight versus displacement test \citep{Sun2016}, at various active lengths of the spring. This provides a calibration curve of stiffness coefficient, $k$ against plate spring length, $l$. We can then adjust the length of the plate spring to obtain the desired value for $k$.

On the other hand, the damping of the system is adjusted using T-shaped aluminium plates fixed at either end of the cylinder endplate, and a pair of neodymium magnets contained in a claw-shaped casing. The further the T-shaped plate is pushed into the opening of the claw, the denser the magnetic field it needs to cut through during motion, thus dissipating more energy. We then calibrate the damping produced at various depths at which the T-shaped plate is pushed into the casing, via free-decay tests of the cylinder in still water. The procedure for conducting free-decay tests are detailed in \citet{Raghavan2007}.

Flow inside the open channel is driven by a $3.728$ kW (5 hp) centrifugal pump, controlled using a voltage controller. The input voltage for the centrifugal pump is calibrated against the centreline velocity of the test section, $750$ mm from the inlet, i.e. mid-length of the test section. We show this schematically in Fig. \ref{fig:keyDimensions}. Here, we define the centreline of the test section as the line $50$ mm from the bottom and $50$ mm from either of the sidewalls of the test section. We placed the cylinder in the same position during experimental runs.

\begin{figure}
  \centering
  \includegraphics[width=0.4\textwidth]{figs/figure4}
  \caption{Side view of the open flow channel, in schematic form. Also, key dimensions of the experimental setup. The acoustic Doppler velocimeter (ADV) is placed at the same location where the cylinder is located during experimental runs.}
  \label{fig:keyDimensions}
\end{figure}

The centreline velocity $U_{\text{cent.}}$ is measured using an acounstic Doppler velocimeter (ADV), sampling at $200$ Hz. The resulting calibration curve is applicable for detemining $U_{\text{cent.}}$ at input voltages $30 < V_{\text{in}} \text{(V)} < 100$. We measured the turbulence intensity along the centreline to be about $5\%$.

We obtained the time history for cylinder displacement, $y$, by using a video camera pointed normal to the cylinder endplate. We placed a visual marker on the endplate, and the motion of the marker captured by the camera is analysed using \textit{Tracker}: a motion analysis tool built on the Open Source Physics Java framework.

To validate our experimental setup, we tuned to the best of our ability our experimental parameters to the values used by \citet{Koide2013} and test whether we can replicate their results. Table \ref{tab:expParameter} summarises the parameters in lieu of that paper.

\begin{table}[width=0.9\linewidth,cols=3,pos=h]
  \caption{Summary of experimental parameters in contrast to those used in the experimental work of \citet{Koide2013}.} \label{tab:expParameter}
\begin{tabular*}{\tblwidth}{@{} LLL@{} }
\toprule
                                           & Current study & \citet{Koide2013}\\
\midrule
Cylinder diameter, $D$ (m)                 & $0.01$        & $0.01$           \\
Cylinder length, $l_{\text{cylinder}}$ (m) & $0.09$        & $0.098$          \\
Strip-plate width (m)                      & $0.01$        & $0.01$           \\
Strip-plate length (m)                     & $0.1$         & $0.1$            \\
Effective mass, $m_{\text{eff.}}$ (kg)     & $0.162$       & $0.174$          \\
Logarithmic damping, $\delta$              & $0.178$       & $0.24$           \\
Scruton number, Sc                         & $9.94$        & $7.74$           \\
System natural frequency, $f_{n}$ (Hz)     & $4.42$        & $4.4$ to $4.79$  \\
\bottomrule
\end{tabular*}
\end{table}

We show a sample of the normalised displacement -- $\ystr = y/D$ -- time series in Fig. \ref{fig:sampTimeHist}. Computing the statistics of $\ystr$ and the normalised cylinder vibration frequency, $\fstr = \fcyl/\fn$ ($\fcyl$ being the vibration frequency of the cylinder), from several runs gave us a value of $\ystr = 0.33 \pm 0.03$ and $\fstr = 1.03 \pm 0.04$. \citet{Koide2013} obtained $\ystr = 0.32$  and $\fstr = 1.09$ under a similar $\ured$ condition. We thus take this fairly successful reproduction of the results of \citet{Koide2013} as an indication of readiness for further data collection.

\begin{figure}
  \centering
  \includegraphics[width=0.41\textwidth]{figs/figure5}
  \caption{A sample of the time history for cylinder displacement from a test run of our experimental setup. the value of $\ured = 22.7$}
  \label{fig:sampTimeHist}
\end{figure}

\section{Numerical setup validation} \label{sec:numSetup}
\subsection{Simple grid independency study} \label{ssec:simpGCI}
Numerical solutions of actual, continuous physical phenomena contain errors, or uncertainties, due to temporal and spatial discretisation. Reliance on the numerical method of investigation puts the responsibility on the user to minimise and justify the magnitude of error introduced in the solution.

While CFD users usually point towards their low Courant-Friedrichs-Lewy number to substantiate their claim of temporal convergence for their numerical solutions, researchers demonstrate the spatial convergence of their solution through either one of these methods. First, by solving the governing equations on several grids, each grid being a finer version of the previous one and showing that the quantities of interest are approximately the constant on all grids tested. One then chooses the mesh with a medium resolution to use in the subsequent data collection \citep{Wu2011,Ding2013,Ding2015a,Ding2019}.

\subsection{Grid independency study via Richardson extrapolation and grid convergence index} \label{ssec:richExtrap}
Like the first, the second method solves the governing equations on successively finer grids. However, instead of arguing that one obtains similar results on all the grids, the investigator checks whether the quantities of interest tend towards value, as one solves the governing equation on successively finer grid resolutions \citep{Richardson1927,Stern2001}. This method, of checking for convergence pays attention not only on the presumed converged value but also on the trend of convergence. Literature that employ this method impose a monotonic convergence condition \citep{Stern2001,MatAli2011,Ali2012,Maruai2018} on their quantities of interest, adding an extra layer of confidence in the final form of heir spatial discretisation.

Additionally, this method allows for a quantitative description of the degree of convergence through the grid convergence index (GCI). Let $f_{1},f_{2},f_{3},\dots,\fn$ denote the quantity of interest obtained from several grids. A larger subscript indicates a coarser grid, this $f_{1}$ denotes the finest while $\fn$ denotes the coarsest grid. Let the difference between successive solutions be $\epsilon_{2,1},\epsilon_{3,2},\epsilon_{4,3},\dots,\epsilon_{n,n-1}$, where $\epsilon_{2,1} = f_{2} - f_{1}$, $\epsilon_{3,2} = f_{3} - f_{2}$ and so on. Then, the GCI is defined as

\begin{equation}
  \text{GCI}_{i+1,i} = F_{s} \frac{\left |\epsilon_{i+1,i} \right |}{f_{i} \left ( r^{p} - 1 \right )} \times 100\%,
  \label{eq:gci}
\end{equation}

\noindent where $F_{s}$, $f_{i}$ and $r^{p}$ denotes the safety factor $\left ( = 1.25 \right )$, quantity of interest and the refinement ratio, $r$, between successive grids raised to the order of accuracu of the series of solution, $p$. We refer the reader to \citet{Stern2001,Langley2018} for a more detailed discussion on $r^{p}$.

We can estimate what the solution approaches as the grid size approaches zero by using the $\text{p}^{\text{th}}$ method. Briefly, we compute the generalised Richardson extrapolation of the quantity of interest as follows.

\begin{equation}
  \fre = f_{1} + \frac{f_{1} - f_{2}}{\rp - 1},
  \label{eq:richardsonExtrapolation}
\end{equation}

\noindent where $\fre$ is the Richardson extrapolation of the quantity of interest. Using $\fre$ to estimate the limit of the monotonically convergent series of $f_{i}$, we can determine the percentage difference of our solution on our finest grid from this limit as

\begin{equation}
  E_{i} = \frac{f_{i} - \fre}{\fre} \times 100\%.
  \label{eq:percentageDifference}
\end{equation}

Table \ref{tab:gridIndependency} summarises the result of our grid independency study for the SVIV reduced velocity of $\ured = 22.7$. We identified three quantities central to the investigation of fluid-structure phenomena, especially the flow-induced vibration of a circular cylinder. They are the vibration amplitude, vibration frequency and lift coefficient of the cylinder. We solve the governing equations on three grids which are numbered $1$ for the finest, $2$ for the medium and $3$ for the coarsest, shown in Fig. \ref{fig:convergenceStudy}. If we let $v_{i}$ be the volume of the $i^{\text{th}}$ cell in the grid, then, the average cell size is


\begin{figure}
  \centering
  \begin{subfigure}[h]{0.3\textwidth}
    \includegraphics[width=\textwidth]{figs/figure6a}
    \caption{Coarse}
    \label{fig:coarseMesh}
  \end{subfigure}

  \begin{subfigure}[h]{0.3\textwidth}
    \includegraphics[width=\textwidth]{figs/figure6b}
    \caption{Medium}
    \label{fig:mediumMesh}
  \end{subfigure}

  \begin{subfigure}[h]{0.3\textwidth}
    \includegraphics[width=\textwidth]{figs/figure6c}
    \caption{Fine}
    \label{fig:fineMesh}
  \end{subfigure}

  \caption{Three meshes used in the grid convergence study. Figs. \ref{fig:coarseMesh}, \ref{fig:mediumMesh} and \ref{fig:fineMesh} show the coarse, medium and fine meshes viewed perpendicular to three main viewing positions: from the inlet, the top and the front, which is looking directly at the cylinder end.} \label{fig:convergenceStudy}
\end{figure}

\begin{equation}
  h = \frac{1}{N} \left [ \sum_{i=1}^{N} v_{i} \right ]^{1/3},
  \label{eq:averageCellSize}
\end{equation}

\noindent and the normalised average cell size is hence 


\begin{equation}
  h/D = \frac{1}{ND} \left [ \sum_{i=1}^{N} v_{i} \right ]^{1/3}.
  \label{eq:normAveCellSize}
\end{equation}

Both $\yrms$ and $\clrms$ starts at an initial value smaller than their Richardson extrapolations, $\fre$, before approaching it as we decrease the average cell size, $h$. This similar trend can perhaps be attributed to the causal relationship between the lift coefficient and vibration amplitude. The lift drives and sustains the vibration, hence a small lift produces a small vibration, and when the lift amplitude becomes higher, so too does the vibration amplitude. The vibration frequency, on the other hand, starts at a value larger than its $\fre$ before approaching $\fre$.

The quantity $\clrms$ experiences the most significant drop in GCI as we refine the grid. The GCI is close to one-third $\left ( 30.92\% \right )$ as we refine the grid from coarse to medium with a refinement ratio of $1.376$. The refinement ratio is calculated by dividing the number of cells in one grid with the next one down the refinement line. Following the grid numbering convention explained previously, dividing the number of cells in the fine grid (grid 1) with the number of cells in the medium grid (grid 2) gives us the refinement ratio from medium to fine, or $r_{2,1}$. Similarly, dividing the number of cells in the medium grid (grid 2) with the number of cells in the coarse grid (grid 3) gives us the refinement ratio from coarse to medium, or $r_{3,2}$. We can generalise this to $n-$number of grids as follows.

\begin{equation}
  r_{i+1,i} = \frac{S_{\text{grid},i+1}}{S_{\text{grid},i}},
  \label{eq:refinementRatio}
\end{equation}

\noindent where $S_{\text{grid},i}$ denotes the total number of cells in the $i^{\text{th}}$ grid. The GCI of $\clrms$ drops further to $1.63\%$ as the mesh is refined more with a refinement ratio of $1.304$.

The GCI for $\yrms$ also drops by one order of magnitude as can be seen by comparing $\text{GCI}_{3,2}$ with $\text{GCI}_{2,1}$. Again, this similar trend of improvement points to the causal relationship between lift and displacement of the cylinder. The GCI for $\fstr$, however, drops by approximately a factor of $6$ instead of one order of magnitude, unlike the GCIs of $\yrms$ and $\clrms$.

\begin{table}[width=0.9\linewidth,cols=4,pos=h]
  \caption{Summary of grid independency study.} \label{tab:gridIndependency}
\begin{tabular*}{\tblwidth}{@{} LLLL@{} }
\toprule
Parameter/ metric                                                       & $\clrms$       & $\yrms = \ystr/D$ & $\fstr = \fcyl / \fn$ \\
\midrule
$\fre$                                                                  & $0.262$        & $0.369$           & $0.969$               \\
$f_{1}$                                                                 & $0.2598$       & $0.3687$          & $0.9695$              \\
$f_{2}$                                                                 & $0.2430$       & $0.3588$          & $0.9740$              \\
$f_{3}$                                                                 & $0.0805$       & $0.2374$          & $1.0220$              \\
$\left | \epsilon_{2,1} \right |$                                       & $0.02$         & $0.01$            & $0.004$               \\
$\left | \epsilon_{2,1} \right |$                                       & $0.16$         & $0.12$            & $0.48$                \\
$R = \left | \epsilon_{2,1} \right | / \left | \epsilon_{2,1} \right |$ & $0.10$         & $0.08$            & $0.094$               \\
$\text{GCI}_{3,2}$                                                      & $30.92$        & $6.00$            & $0.64$                \\  
$\text{GCI}_{3,2}$                                                      & $1.63$         & $0.52$            & $0.10$                \\
\bottomrule
\end{tabular*}
\end{table}

We provide visual representations of the convergent $\clrms$, $\yrms$ and $\fstr$ series in Figs. \ref{fig:yrmsGCI}, \ref{fig:fstrGCI} and \ref{fig:clrmsGCI}. Note how the quantity of interest is very close to its Richardson extrapolation at the fine grid (grid 1) for all $\clrms$, $\yrms$ and $\fstr$. This implies that the fine grid already provides adequate spatial discretisation for the problem we are studying, and further refinements, while able to nudge our solutions even closer to the limit that is the Richardson extrapolation, may not be optimal in terms of usage of computational resources. Values of $\yrms$ and $\fstr$ at the fine grid already fall within experimental uncertainty as evidenced by our measurement in \S \ref{ssec:openFlowExp} and the work by \citet{Koide2013}. Hence, all succeeding numerical data are gathered from the fine grid.


\begin{figure}
  \centering
  \includegraphics[width=0.39\textwidth]{figs/figure7}
  \caption{The convergence diagram for $\yrms$. Fig. \ref{fig:yrmsGCI}a shows how $\yrms$ converges close to the Richardson extrapolation value while Fig. \ref{fig:yrmsGCI}b shows how the error (difference between the value obtained from a particular mesh and the Richardson extrapolation) decreases with decreasing grid spacing.} \label{fig:yrmsGCI}
\end{figure}

\begin{figure}
  \centering
  \includegraphics[width=0.4\textwidth]{figs/figure8}
  \caption{The convergence diagram for $\fstr$. Fig. \ref{fig:fstrGCI}a shows how $\fstr$ convergens close to the Richardson extrapolation value while Fig. \ref{fig:fstrGCI}b shows how the error (difference between the value obtained from a particular mesh and the Richardson extrapolation) decreases with decreasing grid spacing.} \label{fig:fstrGCI}
\end{figure}

\begin{figure}
  \centering
  \includegraphics[width=0.43\textwidth]{figs/figure9}
  \caption{The convergence diagram for $\clrms$. Fig. \ref{fig:clrmsGCI}a shows how $\clrms$ converges close to the Richardson extrapolation value while Fig. \ref{fig:clrmsGCI}b shows how the error (difference between the value obtained from a particular mesh and the Richardson extrapolation) decreases with decreasing grid spacing.} \label{fig:clrmsGCI}
\end{figure}

\section{Single plate amplitude and frequency response} \label{sec:singPlateResp}
\subsection{Amplitude response} \label{ssec:ampResp}
We compared our experiment and numerical results with those from \citet{Koide2013} and \citet{Nguyen2012} in Fig. \ref{fig:ampFreqComp}. Figure \ref{fig:ampFreqComp}a shows the amplitude response of our single plate experiment and simulation. We use the root-mean-square value of the cylinder displacement to represent the amplitude responses instead of the maximum displacement. The reason for this is twofold: first, using  $\yrms$  facilitates comparison of data with \citet{Nguyen2012} and \citet{Koide2013}, who also used  $\yrms$  in their work. Second, because the cylinder displacement is an intermediate quantity for the estimation harnessed power \citep{Maruai2017,Maruai2018}, the usage of root-mean-square of cylinder displacement gives a preview of mean harnessed power, once the vibration is converted into alternating current.

There is virtually no vibration for both our experiment and simulation when  $\ured < 18$, except for a small peak close to $0.1$ at  $\ured \approx 7$. We attribute this peak to the upper branch of KVIV, which still exists, although suppressed due to the cruciform configuration of the system \citep{Shirakashi1989,Nguyen2012}. However, when  $\ured$ exceeds 18, we observe a sudden jump in  $\ured$ right up to about 0.4, for both our experiment and simulation. This we attribute to the formation of the streamwise vortices that drive SVIV.

After the inception of SVIV, the value for  $\yrms$ drops down to approximately 0.3, before recovering to a value that is close to what was observed by \citet{Nguyen2012} and \citet{Koide2013}. This sudden jump followed by a gradual drop and a gradual rise in  $\yrms$ was not found in the works of \citet{Nguyen2012} nor \citet{Koide2013}, even though their experimental parameters are reasonably close to what we use in both our experiment and simulation.


\begin{figure}
  \centering
  \includegraphics[width=0.4\textwidth]{figs/figure10}
  \caption{The amplitude and frequency response of our cruciform system, in lieu of results from \citet{Nguyen2012,Koide2013}. Fig. \ref{fig:ampFreqComp}a shows the amplitude response using $\yrms$, Fig. \ref{fig:ampFreqComp}b the frequency response using $\fstr$ and Fig. \ref{fig:ampFreqComp}c also the frequency response, but using the Strouhal number of vibration.} \label{fig:ampFreqComp}
\end{figure}

We, therefore, attribute this difference to the higher turbulence level set in our work. The turbulence level in the works of \citet{Nguyen2012}, for example, was  $<2.8\%$ throughout their range of Reynolds number. Instead, the initial turbulence level in our setup, both experimental and numerical, is approximately double that value. Because of this, the turbulence amplification due to the onset of streamwise vortices  \citep{Zhao2018a} --- especially for a circular cylinder-strip plate cruciform \citep{Koide2017} --- is also higher compared to the experiments of \citet{Nguyen2012} and \citet{Koide2013}. This higher compound turbulence warps the dominant vortical structure and introduces an increasing amount of intermittency to the lift signal, and by extension, to the displacement time history of the cylinder.

One can simply inspect the error bars within  $18 < \ured < 23$ in Fig. \ref{fig:ampFreqComp}a to verify the greater sample dispersion within that range of  $\ured$. This intermittency ultimately vanishes as the dominant vortical structures become sufficiently stable to retain enough periodicity in its formation. Our numerical results also seem to support this argument, as evidenced by the time history of  $\ured$ within $18 < \ured <30$ in Fig. 11. There exists a distinct increase in intermittency for the time histories in Fig. \ref{fig:unstableSVIV}, that disappears once  $\ured > 23$ as can be seen in Fig. \ref{fig:stableSVIV}.

\begin{figure}
  \centering
  \begin{subfigure}[h]{0.49\textwidth}
    \includegraphics[width=\textwidth]{figs/figure11a}
    \caption{}
    \label{fig:unstableSVIV}
  \end{subfigure}

  \begin{subfigure}[h]{0.49\textwidth}
    \includegraphics[width=\textwidth]{figs/figure11b}
    \caption{}
    \label{fig:stableSVIV}
  \end{subfigure}

  \caption{The time series of cylinder displacement between $18 < \ured < 20$. Fig. \ref{fig:unstableSVIV} groups the cylinder displacement signal between $18 < \ured < 23$, where there seems to be an increase in intermittency in the displacement signal, while Fig. \ref{fig:stableSVIV} groups the cylinder displacment signal between $25 \leq \ured < 30$, where the intermittency in the displacment signal vanishes.} \label{fig:cylDispSignal}
\end{figure}

We see these as grounds for further study on streamwise vortex shedding onset, perhaps from the perspective of transition from convective to absolute instability. However, such studies are more commonly done under low Reynolds number conditions \citep{Wang2019,Li2019} to ease the isolation of the phenomenon and is therefore out of the scope of this study.

\subsection{Frequency response} \label{ssec:freResp}
Figure \ref{fig:ampFreqComp}b compares the frequency responses of our experiment and numerical results with those in \citet{Nguyen2012} and \citet{Koide2013}. We use the normalised frequency  $\fstr$ in Fig. \ref{fig:ampFreqComp}b and the vibration Strouhal number in Fig. \ref{fig:ampFreqComp}c to aid comparison between the results. In our experiments, the value for  $\fstr$ always fall close to unity. However, if we inspect the size of the error bars, we observed a range of  $\ured$ where there exists a higher degree of variance in the sample measurements between  $13 < \ured < 20$. The reason for this lies in  $13 < \ured < 20$ coinciding with the desynchronization region of the KVIV regime up to  $\ured < 18$, and then overlaps with the intermittent vibration regime up to  $\ured < 20$. Within these two regimes, the cylinder displacement time history --- from which  $\fstr$ is calculated --- varies considerably in amplitude and periodicity, resulting in larger error bars. In Fig. \ref{fig:ampFreqComp}c we can see the overall trend being more similar to the results of \citet{Koide2013} rather than \citet{Nguyen2012}, which is likely due to a higher similarity between our experimental setup with that of \citet{Koide2013}, most striking in terms of the gap ratio  $G = g/D$, which is the same.

Our numerical results exhibit a significantly different trend, but only up to  $\ured < 17$. We observe in Fig. \ref{fig:ampFreqComp}b that the vibration frequency of the cylinder increases linearly, even past  $\ured = 7$, which is the upper branch of the KVIV regime. Converting  $\fstr$ into Strouhal number reveals that the cylinder is vibrating close to the Karman frequency of the system. The Karman frequency of a smooth, fixed circular cylinder refers to the shedding frequency of Karman vortices in its wake. An empirical relationship with Reynolds number exists for  $250 < \re < 2 \times 10^{5}$, which is the following \citet{Blevins1990}.

\begin{equation}
  \st = 0.198 \left( 1 - \frac{19.7}{\re} \right)
  \label{eq:karmanSheddingFreq}
\end{equation}

The values we get using Eq. \ref{eq:karmanSheddingFreq} are nearly constant about $0.19$ for  $\ured < 15$. The slight discrepancy from our Strouhal number mean ( $\approx 0.16$) in the   $\ured < 15$ range can be ascribed to us studying a cruciform structure instead of the smooth circular cylinder upon which Eq. \ref{eq:karmanSheddingFreq} was originally based \citep{Blevins1990}.

The discrepancies found especially in Fig. \ref{fig:ampFreqComp}b most probably stem from the same reasons explained by \citet{Nguyen2012}. The lowest  $\yrms$ recorded in our simulation within  $7 < \ured < 15$ was in the order of $10^{-5}$ \si{\metre} (10 microns). A numerical study has no problem recording vibration of this order as the precision of the numerical solution is only limited by the processor architecture. Experimental work, however, requires not only the sensitivity but also the isolation from the background noise that forces the cylinder to vibrate close to the natural frequency of the system  $\fn$ \citep{Nguyen2012}, which consequently overpowers this minimal amplitude vibration. Once streamwise vortices form, however, their shedding and cylinder vibration synchronises close to $\fn$, thus locking the normalised vibration frequency back to  $\fstr \approx 1$.


\section{Temporal evolution of the lift coefficient} \label{sec:tempEvo}
Alternating lift drives the cylinder vibration during VIV. Despite this central position in determining the temporal stability of the amplitude and frequency responses, most studies in the SVIV literature dealt with the lift (coefficient) as if it is only a function of flow velocity, $U$ or reduced velocity, $\ured$ \citep{Kawabata2013,Koide2013,Hemsuwan2018b}. We believe that parties interested in the quality of power harnessed from flow around a cruciform should give similar attention to the transient nature of SVIV as they did for the global characteristics of the flow such as the root mean squares of cylinder displacement, lift coefficient, and dominant frequency through fast Fourier transform (FFT). We think that this is especially the case for the lift signal, to better gauge the room for improvement in future iterations of the system.

\subsection{Ensemble empirical mode decomposition and Hilbert transform} \label{ssec:eemd}
To obtain a clearer picture of the temporal characteristics of the lift and cylinder displacement signals, we decided to employ the ensemble empirical mode decomposition (EEMD) method \citep{Huang1998,Wu2008} on the signals, and compute their instantaneous phase lag, frequency and amplitude using the Hilbert transform.

The Hilbert transform (HT) has been used in the past to study the instantaneous phase and frequencies of KVIV \citep{Khalak1999}. However, the signal must be monochromatic if we are to obtain a physically meaningful result after applying HT. EEMD is a way to pre-process the signal and get components that (1) have zero mean, and (2) have an equal number of extrema and zero crossings, or they differ only by one. Functions that fulfil these criteria are called intrinsic mode functions (IMF), and they guarantee a physically meaningful result to HT \citep{Gumelar2019,Zhou2019}. Unlike Fourier transform, which is an analytical method of signal decomposition based on circular functions in the complex plane, EEMD is algorithmic, and the processes undertaken can be summarised as follows.

Produce 150 white noise signals of length equal to the original signal and amplitude equal to 0.2 of the standard deviation of the original signal. Then, add to the set of white noises the original signal -- creating 150 variations of the original signal. Following that, we apply the empirical mode decomposition (EMD) algorithm on each of the 150 signals. The EMD algorithm is summarised below.

\begin{enumerate} \label{enumerate:emd}
  \item Construct the envelope of the signal by connecting all maxima/minima with cubic splines. \label{enum:envelope}
  \item Find the local mean of the envelope for the span of the data. \label{enum:localMean}
  \item Find the difference between the local mean and the original data. \label{enum:difference}
  \item Repeat steps \ref{enum:envelope} and \ref{enum:localMean} on the difference in \ref{enum:difference} for ten times \citep{Wu2008}.
\end{enumerate}

The steps above produce a set of intrinsic mode functions or IMFs for each of the 150 variations of the original signal. Then, we average the first IMF component from each of the decomposed original signal variations, to obtain the first EEMD IMF $C_{1}$ of the original signal. We do the same for the second, third, until the $n^{\text{th}}$ component for each of the 150 original signal variations, thus obtaining $C_{2},C_{3},\dots,C_{n}$.

To compute the phase lag between lift coefficient Cl and normalised cylinder displacement  $\ystr$, we select the component with the highest correlation to the original signal, to represent the original signal. The phase lag, instantaneous frequency, and instantaneous amplitude of the original signal is subsequently computed by taking the constructing an analitical signal $z \left( t \right)$ from $C_{1},C_{2},\dots,C_{n}$ by computing the Hilbert transform of the IMF, $H_{i}$,

\begin{equation}
  H_{i} \left( t \right) = \frac{1}{\pi} \text{PV} \int\limits_{}^{\infty} \frac{C_{i} \left( \tau \right)}{t - \tau} d\tau,
  \label{eq:hilbertTransform}
\end{equation}

\noindent where PV denotes the Cauchy principal value, and then constructing the analitical signal as follows.
\begin{equation}
  z \left( t \right) = C_{i} \left( t \right) + i H_{i} \left( t \right)
  \label{eq:analiticalSignal}
\end{equation}

\noindent Note that $i$ in Eq. \ref{eq:analiticalSignal} is the complex number.

We refer the reader interested in the details of EEMD and Hilbert transform, also collectively known as the Hilbert-Huang transform (HHT), to the following excellent texts on the subject \citep{Huang2005,Huang2014}.
\subsection{Phase lag in the KVIV regime $\left ( \ured < 14 \right )$} \label{ssec:phaseLag}
At reduced velocities  $\ured = 2.3$ and 4.5, the phase lags  $\phi$ (deg.) between Cl and  $\ured$ are practically zero. The characteristic IMFs of Cl and  $\ystr$ at $\ured = 4.5$ exemplifies this trend, as showcased in Fig. \ref{fig:tempAnalysisKVIV}. The characteristic IMFs are the EEMD components of Cl and  $\ystr$ that has the highest correlation with the original $\ystr$ signal. The trend that one notices in Fig. \ref{fig:tempAnalysisKVIV} is similar to what was observed in \citet{Khalak1999}, a study that also employs the Hilbert transform to obtain the instantaneous phase, albeit without EEMD. Both Cl and  $\ystr$ are in phase with each other and the normalised dominant frequency of the lift coefficient $\fclstr = f_{\text{Cl}}/\fn$ (Fig. \ref{fig:tempAnalysisKVIV}c) falls about one quarter short of the system natural frequency  $\fn$.

\begin{figure}
  \centering
  \includegraphics[width=0.5\textwidth]{figs/figure12}
  \caption{Temporal analysis of the lift coefficient and normalised cylinder displacement signal at $\ured = 4.5$. We show the lift coefficient and normalised cylinder displacement signal side by side in Fig. \ref{fig:tempAnalysisKVIV}a, present the temporal evolution of the phase lag $\phi$ of Cl in Fig. \ref{fig:tempAnalysisKVIV}b and show the temporal evolution of the instantaneous frequency of the lift coefficient signal in Fig. \ref{fig:tempAnalysisKVIV}c. The blue line in Fig. \ref{fig:tempAnalysisKVIV}a represents the lift coefficient signal, while the black line represents the normalised cylinder displacement.} \label{fig:tempAnalysisKVIV}
\end{figure}

Once we enter the upper branch of KVIV at  $\ured = 6.8$, $\phi$ jumps to approximately 110 deg. This jump in $\phi$ is characteristic of the transition to the upper branches as also observed by \citet{Maruai2018}, among others. Both Cl and $\ystr$ signals are visibly very periodic, and the dominant frequency of Cl, i.e. $\fclstr$, is $\approx 1$, as one can verify in Fig. \ref{fig:tempAnalysisUpper}c.

\begin{figure}
  \centering
  \includegraphics[width=0.5\textwidth]{figs/figure13}
  \caption{Temporal analysis of the lift coefficient and normalised cylinder displacement signal at $\ured = 6.8$. We show the lift coefficient and normalised cylinder displacement signal side by side in Fig. \ref{fig:tempAnalysisUpper}a, present the temporal evolution of the phase lag $\phi$ of Cl in Fig. \ref{fig:tempAnalysisUpper}b and show the temporal evolution of the instantaneous frequency of the lift coefficient signal in Fig. \ref{fig:tempAnalysisUpper}c. The blue line in Fig. \ref{fig:tempAnalysisUpper}a represents the lift coefficient signal, while the black line represents the normalised cylinder displacement.} \label{fig:tempAnalysisUpper}
\end{figure}

As we increase $\ured$ even further up to $\ured < 14$, we see a similar trend for all $\ured = 9.1, 11.4, 13.6$ examined: the signal of Cl and $\ystr$ are both qualitatively very periodic, the phase lag is very close to $180$ deg., and the dominant Cl frequency increases linearly in a manner that the Strouhal number of Cl is always $\approx 0.16$ on average. We present a sample of the (1) Cl and  $\ystr$ signals, (2) $\phi$, and (3) $\fclstr$ in the $6.8 < \ured < 14$ range in Figs. \ref{fig:tempAnalysisLower}a, \ref{fig:tempAnalysisLower}b and \ref{fig:tempAnalysisLower}c respectively. The sample is taken from the numerical results at $\ured = 13.6$, and it is characteristic of a KVIV system in the lower branch.

\begin{figure}
  \centering
  \includegraphics[width=0.5\textwidth]{figs/figure14}
  \caption{Temporal analysis of the lift coefficient and normalised cylinder displacment signal at $\ured = 13.6$. We show the lift coefficient and normalised cylinder displacment signal side by side in Fig. \ref{fig:tempAnalysisLower}a, present the temporal evolution of the phase lag $\phi$ of Cl in Fig. \ref{fig:tempAnalysisLower}b and show the temporal evolution of the instantaneous frequency of the lift coefficient signal in Fig. \ref{fig:tempAnalysisLower}c. The blue line in Fig. \ref{fig:tempAnalysisLower}a represents the lift coefficient signal, while the black line represents the normalised cylinder displacement.} \label{fig:tempAnalysisLower}
\end{figure}

\subsection{Transition to SVIV $\left (15.9 < \ured < 18.2 \right )$} \label{ssec:transSVIV}
Previously in the $\ured < 14$ regime, we observed that the temporal profile of both Cl and  $\ystr$ are very similar to each other, except that Cl leads $\ystr$ by a certain amount. This similarity in profile supports the assertion that the vibration within $\ured < 14$ is driven exclusively by the shedding of Karman vortices, which brings the onset of the alternating lift. By extension, one might expect a similar profile between Cl and $\ystr$ even when streamwise vortices drive the vibration. However, this does not seem to be the case.

Once we reach $\ured = 15.9$, we observe that it has become difficult to argue that the profile of $\ystr$ is just a lagged version of the profile of Cl. This is shown in Fig. \ref{fig:tempEvoCompare}a, with the enlarged version in Fig. \ref{fig:tempEvoCompare}b. The profile of Cl looks like the result of several superimposed signals, which one can almost distinguish from the presence of two types of maxima at two different amplitude heights. We put a red dashed line and a red dashed-dot line in Fig. \ref{fig:tempEvoCompare}b as visual cues indicating the two amplitude heights. Decomposing the lift coefficient signal using EEMD reveals partial evidence supporting the superimposed (compound) signal hypothesis.

\begin{figure}
  \centering
  \includegraphics[width=0.4\textwidth]{figs/figure15}
  \caption{Temporal evolution of $\ystr$ and Cl at $\ured 15.9$. Fig. \ref{fig:tempEvoCompare}b shows an enlarged view of Fig. \ref{fig:tempEvoCompare}a. We can barely spot semblance of two signals with different amplitudes superimposed in the Cl signal in Fig. \ref{fig:tempEvoCompare}b.} \label{fig:tempEvoCompare}
\end{figure}

Once we have decomposed the signal using EEMD, we replot Fig. \ref{fig:tempEvoCompare}a using the component of Cl with the highest correlation to the original  $\ystr$ signal and present the comparison in Fig. \ref{fig:tempAnalysisTransition}a. To represent  $\ystr$ in Fig. \ref{fig:tempAnalysisTransition}a, we again chose its IMF component with the highest correlation to the original  $\ystr$ signal, as we have done in Figs. \ref{fig:tempAnalysisKVIV}, \ref{fig:tempAnalysisUpper}, and \ref{fig:tempAnalysisLower}. One can clearly see that the part of Cl signal responsible for driving the vibration at  $\ured = 15.9$ is embedded in the original Cl signal, and decomposition via EEMD managed to recover this signal whose profile is indeed similar to the profile of the characteristic IMF of $\ystr$, except that it leads $\ystr$ on average by approximately 150 deg. (Fig. \ref{fig:tempAnalysisTransition}b). This decline from $\phi \approx 180$ deg. at reduced velocities $6.8 < \ured < 14$, to $\phi \approx 150$ deg. at $\ured = 15.9$ is quite sizeable, suggesting a fundamental change in flow dynamics, particularly in terms of vortical structure.

\begin{figure}
  \centering
  \includegraphics[width=0.45\textwidth]{figs/figure16}
  \caption{Temporal analysis of the lift component that has the highest correlation to the original (normalised) cylinder displacment signal, $\cclystr$, and the normalised cylinder displacement signal at $\ured = 15.9.$ The component was obtained by decomposing the lift coefficient signal using EEMD. We show $\cclystr$ and $\ystr$ signal side by side in Fig. \ref{fig:tempAnalysisTransition}a, present the temporal evolution of the phase lag $\phi$ of $\cclystr$ in Fig. \ref{fig:tempAnalysisTransition}b and show the temporal ecolution of the instantaneous frequency of the $\cclystr$ in Fig. \ref{fig:tempAnalysisTransition}c. The blue line in Fig. \ref{fig:tempAnalysisTransition}c represents the lift coefficient component signal, while the black line represents the normalised cylinder displacement.}
  \label{fig:tempAnalysisTransition}
\end{figure}

Inspecting the HHT spectrogram in Fig. \ref{fig:tempAnalysisTransition}c reveals two dominant bands in the frequency domain. The first one, marked with a white continuous rectangular box, is the instantaneous frequency for the IMF component of lift shown in Fig. \ref{fig:tempAnalysisTransition}a, and its mean frequency lies close to the natural frequency of the system ($\fclstr \approx 1$). There is; however, a second band of the frequency with nearly similar amplitude around $\fclstr \approx 3.3$, marked with a white dashed rectangular box. Computing the Strouhal number from this frequency returns a value of $\st = 0.20$, which is very close to the Strouhal number for Karman vortices as predicted by Eq. \ref{eq:karmanSheddingFreq} at the Reynolds number equivalent to $\ured = 15.9$, which is $\re = 7.9 \times 10^{3}$. We thus attribute this second band of frequency as being the footprint left by the shedding of Karman vortices, and the first band as the result of streamwise vortex shedding.

The knowledge that Karman vortices continue to exist and shed from a cruciform structure during SVIV is not new in the literature. However, this is the first time the lift signal from a cruciform structure undergoing SVIV has been subjected to EEMD, revealing the signature of the two dominant vortical structures regulating the flow around the cruciform. Although the magnitude of the instantaneous frequency due to Karman vortex is comparable to the streamwise vortex (sometimes even bigger), the reason why the cylinder resists locking into its frequency is perhaps that its frequency too distant from the natural frequency of the system $\fn$. The shedding frequency of the streamwise vortex is much closer to $\fn$ and is thus preferred by the cylinder.

We consider the transition to SVIV to be complete at $\ured = 18.2$, when the mean phase lag $\phi$ drops further to $\approx 20$ deg. Figure \ref{fig:tempAnalysisStableInitialBranch}a and \ref{fig:tempAnalysisStableInitialBranch}b documents this observation. The phase lag is observed to slip through 360 deg. At certain portions of the characteristic Cl profile where there are slight distortions in the periodicity of the IMF. The slipping through 360 deg. was also observed by \citet{Khalak1999} in their work on KVIV, which highlights the quasi-periodic nature of the signal being analysed. There, the slip appeared in \citet{Khalak1999} at the initial branch of KVIV. It may be the case that the overall low value of $\phi \approx 20$ deg. at $\ured = 18.2$, coupled with the presence of $\phi$ slippage is suggesting the possibility of $\ured = 18.2$ being the initial branch for SVIV. We could not foresee this point brought up if the original Cl signal is not decomposed beforehand, implying the utility of EEMD in studying fluid-structure interactions with multiple dominant flow structures.

\begin{figure}
  \centering
  \includegraphics[width=0.47\textwidth]{figs/figure17}
  \caption{Temporal analysis of the lift coefficient component that has the highest correlation to the original (normalised) cylinder displacement signal, $\cclystr$, and the normalised cylinder displacement signal at $\ured = 18.2$. The component was obtained by decomposing the lift coefficient signal using EEMD. We show $\cclystr$ and $\ystr$ side by side in Fig. \ref{fig:tempAnalysisStableInitialBranch}a, present the temporal evolution of the phase lag $\phi$ of $\cclystr$ in Fig. \ref{fig:tempAnalysisStableInitialBranch}b and show the temporal evolution of the instantaneous frequency of the $\cclystr$ in Fig. \ref{fig:tempAnalysisStableInitialBranch}c. The blue line in Fig. \ref{fig:tempAnalysisStableInitialBranch}a represents the lift coefficient component signal, while the black line represents the normalised cylinder displacement.}
  \label{fig:tempAnalysisStableInitialBranch}
\end{figure}

\subsection{The stable SVIV regime $\left ( \ured > 20 \right )$} \label{ssec:svivRegime}
As $\ured$ is increased to 20.5, we can see a jump in $\phi$ from a mean value of approximately 20 deg. to about 120 deg., shown in Fig. \ref{fig:phaseAngle}a. The phase slippage discussed previously is also observed in this time series subset, indicating the quasi-periodic nature of the lift coefficient signal at this $\ured$. Incidentally, this quasi-periodicity seems to be the norm for the lift signals up to $\ured = 27.3$, as suggested by the phase slippages evident in Figs. \ref{fig:tempAnalysisStableInitialBranch}b, c and d. The slippage only stops once $\ured$ reaches 29.5, suggesting a more periodic behaviour of the lift coefficient compared to its counterparts between $20.5 \leq \ured \leq 27.3$. Although the instantaneous phase between $20.5 \leq \ured \leq 27.3$ implies a quasi-periodic nature, their mean values at each $\ured$ are contained in the narrow region $114 < \phi$ (deg.) $< 135$, as is the value for $\phi$ at $\ured = 29.5$. This observation that the value of $\phi$ is only slowly varying with respect to $\ured$, once $\ured$ increases past 20.5, can be interpreted as the dominant flow structures settling into a stable form that becomes more resilient against external excitations. Based on this feature, it seems appropriate to classify $20.5 \leq \ured \leq 29.5$ as the upper branch of SVIV.

\begin{figure}
  \centering
  \includegraphics[width=0.45\textwidth]{figs/figure18}
  \caption{The instantaneous phase lag $\phi$ of the dominant component of the normalised cylinder displacement signal ($\ystr$) against $\cclystr$ in the range $20 < \ured < 30$. See Fig. \ref{fig:tempAnalysisStableInitialBranch} for the definition of $\cclystr$.}
  \label{fig:phaseAngle}
\end{figure}

Figure \ref{fig:phaseAngleRegime} summarises our findings thus far, with respect to our analysis of the Cl time series, specifically the ensemble average value of  $\phi$, denoted as $\phim$. The region an indicates the initial branch of  KVIV, where  $\phim$ is close to zero. Region B denotes the upper/lower branch of  KVIV, where the system experiences a jump from  $\phim \approx 0$ to greater than 110 deg. The value of $\phim$ settles very close to 180 deg. towards the end of this upper/lower branch. The HHT spectrograms up to this $\ured$ show only one dominant band of $\fclstr$ which is close to the Strouhal frequency of Karman vortex shedding.

\begin{figure}
  \centering
  \includegraphics[width=0.37\textwidth]{figs/figure19}
  \caption{Vibration regimes identified during analyss of $\phi$. To capture the evolution of $\phi$ with respect to $\ured$, a representative value for $\phi$ at each $\ured$ must be selected. We chose to use the mean $\phi$ as the representative value.}
  \label{fig:phaseAngleRegime}
\end{figure}

Then, $\phim$ experiences a slight drop of about one-sixth the value of $\phim$ at the preceding upper/lower branch as we enter region C, marking the start of the transition to the SVIV regime. The emergence of two dominant instantaneous frequency bands for $\fclstr$ further supports this demarcation. One of the dominant $\fclstr$ band has a value close to unity, and the other has a value close to the shedding frequency of Karman vortex for a fixed, isolated circular cylinder at the same Reynolds number. The system then undergoes a more sudden drop to $\phim \approx 20$ deg. at $\ured = 18.2$. Inspecting the temporal evolution of $\phi$ revealed the quasi-periodic nature of Cl at this $\ured$, which is analogous to the KVIV initial branch studied by \citet{Khalak1999}, prompting us to assign the region up to $\ured = 20.5$ as the initial branch of SVIV (region D).

Finally, in region E, we observe another jump in $\phim$ from $\phim \approx 20$ deg. to approximately 120 deg. as $\ured > 20.5$. The Cl signal gradually loses its quasi-periodicity with increasing $\ured$, and the $\phim$ in this region falls within the arguably narrow range of $114 < \phi$ (deg.) $ < 135$, pointing to stabilisation of dominant flow structures. We hence designate region E as the upper branch of SVIV.

\section{Estimation of harnessable power} \label{sec:estimPow}
\subsection{Mathematical model for power estimation} \label{ssec:mathModel}
The mathematical model for harnessable power estimation in this study follows that which had been derived in \citet{Raghavanetal2007}. In these works, the authors mentioned that work done by the oscillating cylinder $\wcl$ during one cycle of oscillation $\tosc$ is as follows.

\begin{equation}
  \wcl = \int_{0}^{\tosc} \left ( \fl \cdot \dot{y} \right ) dt
  \label{eq:workCylinder}
\end{equation}

\noindent where both the lift $\fl$ and cylinder velocity $\dot{y}$ are both functions of time. Through several manipulations and simplifying assumptions \citep{Sun2016}, the power captured by the system can be written, using our parameters, as the fluid power

\begin{equation}
  \pfrms = \frac{1}{2} \rho \pi \cclrms U^{2} \fcyl \yrms D L \sin(\phi),
  \label{eq:rmsFluidPower}
\end{equation}

\noindent or the mechanical power

\begin{equation}
  \pmrms = 8 \pi^{3} \meff \zetatot \left (\yrms \fcyl \right )^{2} \fn.
  \label{eq:rmsMechPower}
\end{equation}

Here, $\pfrms$, $\pmrms$, $L$, $\cclrms$, $\zetatot$ and $\meff$ are the root mean square of fluid power, root mean square of mechanical power, length of the circular cylinder, characteristic root mean square of lift amplitude, total damping coefficient, and the system effective mass respectively. We choose to use root mean square (parameters with subscript RMS) quantities in Eq. \ref{eq:workCylinder} instead of the maximum values like the original authors because that may lead to a misunderstanding that the maximum value is sustained throughout the observation window. This obviously is not always the case in our study, especially once the flow transits to SVIV. Time series analysis of $\ystr\left( t \right)$ and $\text{Cl}\left( t \right)$ in \S\ref{ssec:ampResp} revealed that there is a degree of intermittency in both signals that cannot be overlooked at specific ranges of $\ured$, thus making it better to use the root mean square values instead. Estimation of the root mean square of harnessable power in our opinion makes more sense because it returns a value that is continually approached by the system \textit{over time}, while the maximum, could be a one-off value.

Before presenting the results of our harnessable power estimation following Eqs. \ref{eq:rmsFluidPower} and \ref{eq:rmsMechPower}, let us clarify our method of estimating the root mean square of lift amplitude $\cclrms$. Let $\fl\left( t \right)$ be the lift acting on the cylinder and $y\left( t \right)$ the cylinder displacement time series resulting from that alternating lift. Decomposing $\fl\left( t \right)$ via EEMD yields a finite number $N$ of IMFs which we can summarily write as $F_{L} \left( t \right) = \sum\limits_{}^{N}C_{i} \left( t \right) $. The IMF chosen as the component of lift driving $y \left( t \right) $  is the $C_{i} \left( t \right) $  with the highest correlation with  $ y \left( t \right)  $ , i.e. the component due to streamwise vortex. We then compute the root mean square value of that component of lift, giving us $\cclrms$.

Figure \ref{fig:powerComparison} shows the comparison between power estimated from our experiment and numerical results, with the experimental results of \citet{Nguyen2012} and the direct power measurement of \citet{Koide2013}. Only the value for $\pmrms$ is computed from our experimental results due to the absence of lift data. Our numerical results have both lift and cylinder displacement data, and hence, we calculated both $\pfrms$ and $\pmrms$. We estimated the power from the experimental results of \citet{Nguyen2012} by interpolating missing data points in both their amplitude and frequency responses to compute the value of $\pmrms$ at a given value of $\ured$. The direct power measurement by \citet{Koide2013} was done by connecting the elastic support of the cylinder to a coil. The coil moves with the cylinder, thus creating a relative pistoning motion against a fixed magnet and produces an alternating current.

\begin{figure}
  \centering
  \includegraphics[width=0.4\textwidth]{figs/figure20}
  \caption{Estimated root mean square of mechanical power $\pmrms$, fluid power $\pfrms$, or both, of our experimental and numerical results, compared with results of similar studies in the literature. The fluid power $\pfrms$ is calculated only from the results of our numerical study as the others did not measure lift. The computation of the instantaneous phase lag $\phi$ requires both lift and cylinder displacement signals.}
  \label{fig:powerComparison}
\end{figure}


\begin{figure}
  \centering
  \includegraphics[width=0.47\textwidth]{figs/figure21}
  \caption{The instantaneous frequency of the lift signal between $20 < \ured < 30$. The white, solid box encloses the region where the mean frequency is close to the system natural frequency $\fn$, while the dashed, white box encloses the region where the mean frequency is close to the shedding frequency of Karman vortex at the Reynolds number at which at the simulation is performed. Through visual inspection, we can see how the degree of dispersion in the instantaneous frequency of the ``Karman component'' of lift is about twice that of the ``streamwise component'' of lift.}
  \label{fig:instantLiftFreq}
\end{figure}

We note that the evolution trend of estimated power with respect to $\ured$ is similar between $\pmrms$ from our experiment and simulation, especially in the $\ured$ region immediately after the onset of SVIV. This makes sense since $\pmrms$ is basically a single variable function, the variable being $\yrms$, with the others fixed as we vary $\ured$. The trend observed in $\pmrms$ is thus a scaled version of the trend found in $\yrms$. Nevertheless, besides this region of $18 < \ured < 23$ the trend between all data series compared in Fig. \ref{fig:instantLiftFreq} are relatively similar. This trend is especially the case after $\ured > 23$, where we observe a fairly good agreement between $\pmrms$ and $\pfrms$ computed from our experimental and numerical results with the direct power measurements of \citet{Koide2013} and the estimated $\pmrms$ from the data of \citet{Nguyen2012}. The estimated power in the KVIV regime $\ured < 17$ produces power only in the order of \si{\micro\watt}, which is relatively insignificant in contrast to the magnitude of power produced in the SVIV regime (mW).

\subsection{Possibility for increasing fluid power, $\pfrms$} \label{ssec:possIncrease}
We have seen in Fig. \ref{fig:powerComparison} the similarity in the evolution trend of  $\pmrms$ and $\pfrms$ against $\ured$ of our numerical results with those from \citet{Nguyen2012} and \citet{Koide2013}. However, recall that to represent the amplitude of lift, we used the root mean square amplitude of the component of lift that has the highest correlation with the original cylinder displacement signal $y \left( t \right)$. We did not use the root mean square amplitude of the original lift signal, and yet we obtained $\pfrms$ estimates that are in reasonable agreement not only with its $\pmrms$ counterparts but with the actual measured power of \citet{Koide2013}.

On the one hand, this is an indication that the lift component selected for use in computation is an arguably faithful representation of the force driving the motion of the cylinder. The fact that it is a reasonably good representation also suggests that the motion of the cylinder, once it enters the SVIV regime, is driven only by a component, and not the totality of the lift force. Another significant subset of the lift force is the component whose mean frequency is close to the Karman frequency of vortex shedding, as explained in \S\ref{ssec:transSVIV}. This Karman component of lift has a similar order of magnitude to the streamwise component of lift, as evidenced in Fig. \ref{fig:instantLiftFreq}, and is therefore not negligible. The Karman components are marked with a dashed, white box, and the streamwise components are marked with a solid, white box, following the convention in Figs. \ref{fig:tempAnalysisKVIV}, \ref{fig:tempAnalysisUpper}, \ref{fig:tempAnalysisLower}, \ref{fig:tempAnalysisTransition} and \ref{fig:tempAnalysisStableInitialBranch}. However, the Karman component fails to affect the cylinder vibration like the streamwise component most probably due to the large difference between the mean frequency of the Karman component and the natural frequency of the system, $\fn$.  The streamwise component has a mean frequency close to $\fn$ and is hence able to synchronise with the vibration of the cylinder, producing a sizeable amplitude response.

\begin{figure}
  \centering
  \includegraphics[width=0.43\textwidth]{figs/figure22}
  \caption{Evolution of the root mean square amplitude of two dominant lift components, Karman and streamwise vortices with respect to $\ured$. The region $\ured < 23$ exhibits similar magnitude for both the Karman and streamwise components of lift. On the other hand, the magnitude of amplitude for the Karman component while the region $\ured > 23$ is almost always twice that of the streamwise component.}
  \label{fig:karmanStreamwiseComponents}
\end{figure}

Figure \ref{fig:karmanStreamwiseComponents} shows the \rms{} amplitude of the Karman and streamwise components of lift in the SVIV regime $\ured > 18$. Between $18 < \ured < 23$, the magnitude of the Karman and streamwise components are nearly equal. However, once we exceed $\ured = 23$, Fig. \ref{fig:karmanStreamwiseComponents} shows that the contribution to the \rms{} amplitude of total lift by the Karman component is on average twice the contribution of the streamwise component. Let us assume a hypothetical situation where we can transfer the contribution by the Karman component to the streamwise component of lift. Then, the value for $\cclrms$ in Eq. \ref{eq:rmsFluidPower} will increase close to a factor of 2 when $18 < \ured < 23$, and close to a factor of 3 when $23 < \ured < 30$. This increase in $\cclrms$ will lead to a larger $\pfrms$, if the value of the other parameters in Eq. \ref{eq:rmsFluidPower} are similar. This exercise demonstrates the room for improvement possible for $\pfrms$ in future developments of cruciform energy harvesters.

\section{Conclusions} \label{sec:conclusions}
In this study, we numerically investigated the temporal evolution of the lift coefficient and cylinder displacement signals of an elastically supported cruciform system in the range $1.1 \times 10^{3} < \re < 14.6 \times 10^{3}$, or $\uron < \ured < \urtt$. Our circular cylinder diameter is \SI{10}{\milli\metre} and the natural frequency of the system is \SI{4.4}{\hertz}. Validation of key numerical results was made experimentally in a custom-built open flow channel, using a cruciform system whose parameters were tuned as close as possible to the quantities used in the numerical study.

We observed the amplitude response to reach large magnitudes when the dominant frequency of lift is close to the natural frequency of the system, i.e. $\fn$. This observation explains the maxima in the amplitude response at $\ured = \urth$, which takes place at the upper branch of the KVIV regime, i.e. $\uron \leq \ured \ursi$. The onset of streamwise vortex shedding imposes an additional degree of complexity on the lift coefficient signal, causing it to deviate from the sinusoidal-like signature of lift observable in the KVIV regime. This complexity is, however, not observed in the cylinder displacement signal, which remains relatively similar to a sinusoidal function. Decomposing the lift coefficient signal in the SVIV regime ($\urse \leq \ured \leq \urtt$) using EEMD allows us to see that the complexity of the lift coefficient signal as probably being caused by the superimposition of two dominant components of lift. One due to the shedding of Karman and the other due to the shedding of streamwise vortices. This component of lift has a mean frequency close to $\fn$. Through visual inspection, it is relatively similar to a sinusoidal function. This sinusoidal profile results in a similar pattern in the cylinder displacement signal in the SVIV regime.

Application of the Hilbert-Huang transform on the most dominant component of cylinder displacement -- and the component of lift most correlated to it -- allows for the computation of the instantaneous phase lag between lift and cylinder displacement. The temporal mean of the instantaneous phase lag revealed five ``branches'' of vibration, among which is the initial branch of SVIV at $\ured = \urei$, which has never been identified before in the literature.

Estimation of power from our results show that the root-mean-square mechanical and fluid power computed from our numerical work to be in fairly good agreement with the root-mean-square mechanical power computed from our experiments. There are, however, discrepancies with the trend found in other literature, especially within $\urse \leq \ured \leq \urni$, which is right after the onset of SVIV. These discrepancies are probably due to deviations from the literature in terms of the fluid environment we subject the cruciform system to during data collection (open flow channel v.s. water tunnel, medium flow turbulence v.s. low flow turbulence). Finally, we estimated the upper limit for improvement of the root-mean-square fluid/mechanical power and found that the \rms{} powers can potentially be increased close to a factor of 2 within $18 < \ured < 23$ and close to a factor of 3 when $23 < \ured < 30$. We base the estimation on the ability to minimise the contribution from Karman vortices, while maximising the contribution from streamwise vortices towards the total \rms{} lift amplitude.

%\appendix
%\section{Appendix}
%Appendix sections are coded under \verb+\appendix+.
%
%\verb+\printcredits+ command is used after appendix sections to list 
%author credit taxonomy contribution roles tagged using \verb+\credit+ 
%in frontmatter.
%
%\printcredits

%% Loading bibliography style file
%\bibliographystyle{model1-num-names}
\bibliographystyle{cas-model2-names}

% Loading bibliography database
\bibliography{singlePlateRefs}


%\vskip3pt

%\bio{}
%Author biography without author photo.
%Author biography. Author biography. Author biography.
%Author biography. Author biography. Author biography.
%Author biography. Author biography. Author biography.
%Author biography. Author biography. Author biography.
%Author biography. Author biography. Author biography.
%Author biography. Author biography. Author biography.
%Author biography. Author biography. Author biography.
%Author biography. Author biography. Author biography.
%Author biography. Author biography. Author biography.
%\endbio
%
%\bio{figs/pic1}
%Author biography with author photo.
%Author biography. Author biography. Author biography.
%Author biography. Author biography. Author biography.
%Author biography. Author biography. Author biography.
%Author biography. Author biography. Author biography.
%Author biography. Author biography. Author biography.
%Author biography. Author biography. Author biography.
%Author biography. Author biography. Author biography.
%Author biography. Author biography. Author biography.
%Author biography. Author biography. Author biography.
%\endbio
%
%\bio{figs/pic1}
%Author biography with author photo.
%Author biography. Author biography. Author biography.
%Author biography. Author biography. Author biography.
%Author biography. Author biography. Author biography.
%Author biography. Author biography. Author biography.
%\endbio

\end{document}
